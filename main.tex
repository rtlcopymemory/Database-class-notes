\documentclass[10pt]{article}

\usepackage[italian]{babel}
\selectlanguage{italian}

\usepackage{ulem} % strikethrough

\usepackage{scrextend} % indentation

\usepackage{graphicx} % for pics
\graphicspath{{images/}{../images/}} % source folder for pics

\usepackage{amsmath} % Math

\usepackage{hyperref} % Clickable Table of content
\hypersetup{
    colorlinks,
    citecolor=black,
    filecolor=black,
    linkcolor=black,
    urlcolor=black
}

\usepackage{tabularx} % Tables

\usepackage[margin=1.5in,bottom=1.5in,top=1.5in]{geometry} % changes the margin
% Fancy header and footer with page number on the right
\usepackage{fancyhdr}
\usepackage{lastpage}
\pagestyle{fancy}
\fancyfoot{}
\rfoot{\thepage}

% Algorithms Pseudo-code
\usepackage{algorithm}
\usepackage{algorithmic}

% Enviroment called theorem, resetting every section, printing 'Theorem'
\newtheorem{theorem}{Teorema}[section]
\newtheorem{definition}{Definizione}[section]
\newtheorem{demonstration}{Dimostrazione}[definition]

\def\code#1{\texttt{#1}}

\usepackage{subfiles} % Best loaded last in the preamble

\title{Basi di Dati\\\Large{Informatica}}
\author{Federico Crippa}
\date{Unige\\A.A. 2019/2020}

\begin{document}

\maketitle
\tableofcontents

\newpage
\section*{Disclaimer}
\subfile{sections/disclaimer.tex}

\section*{Credits}
\subfile{sections/credits.tex}
\newpage

\part{Basi di Dati 1}
\section{Modello relazionale}
\subfile{sections/modello.tex}

\newpage
\section{SQL DDL}
\subfile{sections/SQL-DDL.tex}

\section{Algebra Relazionale}
\subfile{sections/AlgebraRelazionale.tex}

\section{Normalizzazione}
\subfile{sections/Normalizzazione.tex}

\newpage
\section{Triggers}
\subfile{sections/Triggers.tex}

\newpage
\section{Progettazione}
\subfile{sections/progettazione.tex}

\newpage
\part{Basi di Dati 2}
\section{Architettura di un DBMS}
\subfile{sections/architettura.tex}

\break
\section{Controllo dell'accesso}
\subfile{sections/accesso.tex}

\break
\section{Strutture di memorizzazione}
\subfile{sections/memorie.tex}

\section{Indici}
\subfile{sections/indici.tex}

\section{Ottimizzazione Logica}
\subfile{sections/ottimizzazioneLogica.tex}

\break
\section{Ottimizzazione Fisica}
\subfile{sections/ottimizzazioneFisica.tex}

\break
\section{Tuning Logico}
\subfile{sections/TuningLogico.tex}

\section{Tuning Fisico}
\subfile{sections/TuningFisico.tex}

\section{Tuning Delle Interrogazioni}
\subfile{sections/TuningQuery.tex}

\section{Transazioni}
\subfile{sections/Transazioni.tex}

\section{Controllo della Concorrenza}
\subfile{sections/Concorrenza.tex}

\section{Gestione del Ripristino}
\subfile{sections/Ripristino.tex}

\end{document}

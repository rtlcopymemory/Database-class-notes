Ha $5 + 1$ operatori base ($5$ operatori $+ 1$ che \`e la ridenominazione).

\subsection{Operatori Unari}
\subsubsection{Proiezione}
$$\Pi_{A, B, ...}(Relazione)$$
Corrisponde al \code{SELECT}\\
Monotono

\subsubsection{Selezione}
$$\sigma_{F}(Relazione) \qquad F = \text{funzione booleana}$$
Corrisponde al \code{WHERE}\\
Monotono

\subsubsection{Ridenominazione}
$$\rho_{vecchi\_nomi \leftarrow nuovi\_nomi}(Relazione)$$
Corrisponde al \code{AS}\\
Monotono

\subsection{Operatori Binari}
\subsubsection{Prodotto cartesiano}
$$R_1 \times R_2 \Rightarrow R$$
Non \`e possibile eseguirlo se le relazioni hanno attributi in comune.\\
Grado $R =$ Grado $R_1 +$ Grado $R_2$\\
Cardinalità $R =$ Cardinalità $R_1 \times$ Cardinalità $R_2$\\
Corrisponde a \code{FROM $R_1$, $R_2$}\\
Commutativo\\
Monotono

\subsubsection{Unione}
$$R_1 \cup R_2 \Rightarrow R$$
Grado $R =$ Grado $R_1 =$ Grado $R_2$\\
Cardinalità $R =$ Cardinalità $R_1 +$ Cardinalità $R_2$\\
Corrisponde a \code{UNION}\\
Commutativo\\
Monotono

\subsubsection{Differenza}
$$R_1 - R_2 \Rightarrow R$$
$$R_1 \backslash R_2 \Rightarrow R$$
Grado $R =$ Grado $R_1 =$ Grado $R_2$\\
Cardinalità $R \leq$ Cardinalità $R_1$\\
Corrisponde a \code{EXCEPT}

\subsection{Operatori aggiuntivi o derivati}
\subsubsection{Intersezione}
$$R_1 \cap R_2 \Rightarrow R$$
$$R_1 \cap R_2 = R_1 - (R_1 - R_2)$$
Grado $R =$ Grado $R_1 =$ Grado $R_2$\\
Corrisponde a \code{INTERSECT}\\
Commutativo

\subsubsection{Theta Join}
$$R_1 \bowtie_{F} R_2 \Rightarrow R$$
$$R_1 \bowtie_{F} R_2 = \sigma_{F}(R_1 \times R_2)$$
Grado $R =$ Grado $R_1 +$ Grado $R_2$\\
Cardinalità $R \leq$ Cardinalità $R_1 \times$ Cardinalità $R_2$\\
Un Join con condizione di Uguaglianza viene definito \textit{Equi-Join}\\
Corrisponde a \code{$R_1$ JOIN $R_2$ ON $F$}

\subsubsection{Join Naturale}
$$R_1 \bowtie R_2 \Rightarrow R$$
Corrisponde a un prodotto cartesiano nel caso in cui non ci siano attributi in comune.\\
Grado $R =$ Grado $R_1 +$ Grado $R_2 -$ numero colonne in comune\\
Cardinalità $R \leq$ Cardinalità $R_1 \times$ Cardinalità $R_2$\\
Corrisponde a \code{NATURAL JOIN}

\subsubsection{Divisione}
$$R_1 \div R_2 \Rightarrow R$$
$$R_1 \div R_2 = \Pi_{D}(R_1) - \Pi_{D}((\Pi_{D}(R_1) \times R_2) - R_1)$$
$D =$ Attributi non in comune.\\
$(\Pi_{D}(R_1) \times R_2) =$ Tuple su $D$ completate in tutti i possibili modi rispetto a $R_2$ (tutte le possibilità)\\
$\Pi_{D}((\Pi_{D}(R_1) \times R_2) - R_1) =$ Tuple su $D$ per cui manca almeno un completamento in $R_2$ \vspace{2mm} \\
Grado $R =$ Grado $R_1 -$ Grado $R_2$\\
Cardinalità $R \leq$ Cardinalità $R_1$\\
Non esiste un operatore preciso in SQL ma si possono usare \code{NOT EXISTS} ristrutturando la query in 2 sotto-query oppure \code{HAVING COUNT}
$$(R \div S) \times S \not= R$$
$$(R \times S) \div S = R$$
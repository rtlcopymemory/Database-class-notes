La normalizzazione esiste per risolvere dei problemi che potrebbero causare perdita di efficienza o difficoltà a mantenere L'Integrità dei dati. Queste anomalie sono:
\begin{itemize}
    \item \textbf{ridondanza}: attributi ripetuti
    \item \textbf{anomalia di aggiornamento}: aggiornare una tupla richiede aggiornarne multiple
    \item \textbf{anomalia di inserimento}: inserire una tupla richiede l'inserimento di tutti gli attributi
    \item \textbf{anomalia di cancellazione}: cancellare tutte le tuple per un solo insieme di attributi vuol dire perdere informazioni anche per gli altri attributi
\end{itemize}

\subsection{Dipendenze Funzionali}
Esiste in $r$ una dipendenza funzionale (FD) da $Y$ a $Z$ se, per ogni coppia di tuple $t_1$ e $t_2$ di $r$ con gli stessi valori su $Y$, risulta che $t_1$ e $t_2$ hanno gli stessi valori anche su $Z$.\\
Notazione:\\
$X$ implica $Y$
$$X \rightarrow Y$$
Sapendo l'attributo $X$ ricaviamo $Y$, oppure $Y$ \`e univocamente ricavabile da $X$.\\
Si dice che una FD \`e \textit{banale} se dato $Y \rightarrow A$, $A$ fa parte di $Y$ (ad esempio $A B \rightarrow A$).\\
Si dice che un insieme di attributi \`e \textbf{chiave} quando:
\begin{itemize}
    \item \`e minimale
    \item da tali attributi, tramite le FD si arriva a tutti gli attributi della tupla
    \item gli attributi non si ripetono mai nella tabella
\end{itemize}
Se non \`e minimale allora \`e \textit{superchiave}

\subsection{BCNF}
Una relazione $r$ \`e in forma normale di Boyce e Codd se, per ogni dipendenza funzionale (non banale) $X \rightarrow Y$ definita su di essa, $X$ contiene una chiave $K$ di $r$.\\
Se una relazione non \`e in BCNF allora va divisa in più relazioni in base alle dipendenze funzionali.

\subsection{Decomposizioni senza perdita}
Una decomposizione si dice senza perdita se facendo il join fra le tabelle ottenute ottengo sempre la tabella di partenza.\\
Una decomposizione dovrebbe sempre soddisfare:
\begin{itemize}
    \item La decomposizione senza perdita, che garantisce la ricostruzione delle informazioni originarie.
    \item La conservazione delle dipendenze, che garantisce il mantenimento dei vincoli di integrità originari
\end{itemize}

\subsection{Terza forma normale (3NF)}
Una relazione $r$ \`e in terza forma normale se, per ogni FD (non banale) $X \rightarrow Y$ definita su $r$, \`e verificata \textbf{almeno} una delle seguenti condizioni:
\begin{itemize}
    \item $X$ contiene una chiave $K$ di $r$
    \item ogni attributo $Y$ \`e contenuto in almeno una chiave di $r$
\end{itemize}
Essendo una forma normale più permissiva della BCNF \`e anche sempre raggiungibile (al contrario della BCNF).
\vspace{6mm} \\
Se la relazione ha una sola chiave allora le due forme normali coincidono.
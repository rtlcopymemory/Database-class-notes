Il \textbf{Tuning} in generale tratta dell'ottimizzazione degli schemi (logici e fisici) a seconda delle richieste di un determinato \textit{workload}.\\
Il \textbf{Tuning logico} in particolare riguarda l'eventuale riscrittura dello schema logico a seconda dei principi della normalizzazione o, a volte, della convenienza in termini di prestazioni.

\subsection{Tuning della normalizzazione}
\textbf{Due relazioni} normalizzate R1(XY) e R2(XZ) sono preferibili a una singola relazione normalizzata ad esse equivalente R(XYZ) se:
\begin{itemize}
    \item le operazioni più frequenti accedono solo XY o XZ
    \item gli attributi Y e Z ammettono valori che occupano molto spazio
\end{itemize}
In questo modo si riduce la dimensione delle relazioni che devono essere accedute.\vspace{2mm} \\
Una \textbf{singola relazione} normalizzata R(XYZ) è meglio di due relazioni normalizzate e ad essa equivalenti R1(XY) e R2(XZ) se:
\begin{itemize}
    \item le operazioni più frequenti richiedono di accedere contemporaneamente XY e Z
\end{itemize}
In questo modo, si evita l’esecuzione di un join.

\subsection{Tuning della denormalizzazione}
Pros:
\begin{itemize}
    \item può migliorare le prestazioni quando attributi che appartengono a diverse relazioni normalizzate devono essere spesso acceduti insieme.
\end{itemize}
Cons:
\begin{itemize}
    \item peggiora le prestazioni in caso di aggiornamenti frequenti.
\end{itemize}
Denormalizzare lo schema a livello logico significa aggiungere allo schema logico nuove relazioni, non normalizzate, corrispondenti al risultato di join frequenti, eseguiti nell’ambito di operazioni contenute nel carico di lavoro.
\begin{itemize}
    \item C'\`e bisogno di Trigger per aggiornare le nuove tabelle ogni volta che le tabelle di base vengono aggiornate.
\end{itemize}
\vspace{4mm}
\`E possibile usare una via di mezzo fra normalizzazione e Denormalizzazione, cioè il meccanismo di \textbf{Viste materializzate}. In postgreSQL:\vspace{2mm}\\
\code{CREATE MATERIALIZED VIEW Name\\
AS <Query>}
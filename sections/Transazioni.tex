Una transazione \`e una sequenza di operazioni di lettura e scrittura sulla base di dati a cui viene garantita un’esecuzione che soddisfa alcune proprietà: correttezza, robustezza, isolamento.

\subsection{Proprietà ACID}
\begin{itemize}
    \item \textbf{Atomicity}: La transazione viene eseguita in maniera atomica. Se un pezzo non viene completato, allora nulla lo \`e.
    \item \textbf{Consistency}: Nessun vincolo di integrità viene violato
    \item \textbf{Isolation}: Ogni transazione esegue indipendentemente dalle altre eseguite allo stesso momento
    \item \textbf{Durability}: Gli effetti di una transazione completata sono persistenti nel tempo
\end{itemize}

\subsection{Transazioni Flat}
Sono le più usate. Si caratterizzano dalle parole chiave:
\begin{itemize}
    \item \code{BEGIN}: Inizia una transazione
    \item \code{COMMIT}: Completa con successo una transazione ("ufficializza" il risultato)
    \item \code{ROLLBACK}: Termina (anticipatamente) la transazione.
\end{itemize}

\subsection{Controllo dei vincoli}
Ci sono 2 modalità di controllo:
\begin{itemize}
    \item \textbf{Valutazione immediata}: Dopo ogni comando viene eseguito il \code{CHECK}. Se il comando viola il vincolo viene annullato.
    \item \textbf{Valutazione differita}: Controllo al termine della transazione.
    \begin{itemize}
        \item \code{DEFERRABLE INITIALLY IMMEDIATE}: Verifica dopo ogni istruzione
        \item \code{DEFERRABLE INITIALLY DEFERRED}: Verifica al termine della transazione
    \end{itemize}
\end{itemize}